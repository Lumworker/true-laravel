//สร้างตัวแปร Usermod โดยเรียกฟังชั่นจากไฟล์ user.php
use App\User as UserMod;//adบบรนทัด18
use App\Model\shop as Shopmod;//สร้างให้shopmodel =shopข้างหน้า
use App\Model\Product as ProductMod;


  
    public function index()
    {
        
        $mods = UserMod::paginate(5);//แก้เพื่อเวลาทำpagination จะให้มีกี่แถวต่อหน้า
        return view('admin.user.lists',compact('mods'));


    }

    public function create()
    {
        return view ('admin.user.create');
        //เรียกcreate.blade.php โดยที่ createดังกล่าวต้อง ประกาศ extends และ section
    }

    public function store(Request $request)
    {
            request()->validate([
                'name' => 'required|min:2|max:50',
                'surname' => 'required|min:2|max:50',
                'mobile' => 'required|numeric',
                'email' => 'required|email|unique:users',
                'password' => 'required|min:6',
                'confirm_password' => 'required|min:6|max:20|same:password',
            ], [
                'name.required' => 'Name is required',
                'name.min' => 'Name must be at least 2 characters.',
                'name.max' => 'Name should not be greater than 50 characters.',
            ]); 
            $mod = new UserMod;
            $mod->email    = $request->email;
            $mod->password = bcrypt($request->password);
            $mod->name     = $request->name;
            $mod->surname  = $request->surname;
            $mod->mobile   = $request->mobile;
            $mod->age      = $request->age;
            $mod->address  = $request->address;
            $mod->city     = $request->city;
            $mod->save();
            
        // เอาnameจากกล่องinput ทีเราพิมใว้ใน create.blade.php
        echo "success .";
    }

    /**
     * Display the specified resource.
     *
     * @param  int  $id
     * @return \Illuminate\Http\Response
     */
    public function show($id)
    {
        //
    }

    /**
     * Show the form for editing the specified resource.
     *
     * @param  int  $id
     * @return \Illuminate\Http\Response
     */
    public function edit($id)
    {
        //
    }

    /**
     * Update the specified resource in storage.
     *
     * @param  \Illuminate\Http\Request  $request
     * @param  int  $id
     * @return \Illuminate\Http\Response
     */
    public function update(Request $request, $id)
    {
        //
    }

    /**
     * Remove the specified resource from storage.
     *
     * @param  int  $id
     * @return \Illuminate\Http\Response
     */
    public function destroy($id)
    {
        //
    }