เป็นการ ส่วนของ UsersControllers.php ที่อ.ให้ลองหลายๆรูปแบบ
    public function index()
    {
        /*
        $data = [
            'name' => 'My Name',
            'surname' => 'My SurName',
            'email' => 'myemail@gmail.com'
        ];
 
         $user = UserMod::find(1);
         $mods = UserMod::all();
 
         return view('template', compact('data', 'user', 'mods'));
         */
 
        // $mods = UserMod::all();
        // return view('template', compact('mods'));//คำสั่งcompactจะนำmods และส่งไปทุกตัว

    

    // $item = [
    //     'item1' => 'My Value1',
    //     'item2' => 'My Value2'
    // ];

    // $results = [
    //     'data' => $data,
    //     'item' => $item
    // ];

    // return view('template', $results);


        // --------------------------------
    //     $data = [ //การส่งแบบarray
    //         'name' => 'My Name',
    //         'surname' => 'My SurName',
    //         'email' => 'myemail@gmail.com'
    //     ];

    //    return view('template', $data);


        /*
        return view('template')
        ->with('name', 'My Name')
        ->with('surname', 'My SurName')
        ->with('email', 'myemail@gmail.com');
            */
        //return view('template')->with('name', 'My Name');ส่งไปหน้า .blade และส่งค่าไปด้วย

        //return view('template');//ไปเรียกหน้า template.blade.php **สำคัญพวกหน้าpageต้องมีblade



        // $shop = ProductMod::find(1)->shop;
        // echo "<br />Product is belong to shop :".$shop->user->name;//produc idเป็น1ไปจอย กับshop ที่ id เป็น1


       /* // $products = \App\Model\Shop::find(1)->products;
        $products = ShopMod::find(1)->products;
        //dd($products);
        foreach ($products as $item) {//สั่งprint $productsจนถึงตัวสุดท้าย
            echo"<br /> product :".$item->name."<br /> Status : ".$item->status ;
        /*        }




        //sql
        // $mods = UserMod::all();
        // $mods = UserMod::where('active', 1) //where active =1
        //                 ->where('name','link','%user2%')
        //        ->orderBy('name', 'desc')
        //        ->take(10)//เอาแค่10อัน
        //        ->get();//ต้องติดgetเมื่อใช้ where
        //------------------------------------------
        // $user = UserMod::find(1);
        // echo "<br> User: ". $user->name;

        // $shop = UserMod::find(1)->shop;
        // echo "<br> Shop Name: ".$shop->name;//จิ่มไปที่shopเลย
       
        // $shop = UserMod::find(1);
        // echo "<br> Shop Name2: ". $user->shop->name;//สามารถเขียนแบบนี้ก็ได้แบบ$shopข้างบนเช่นกัน

        // $shop = UserMod::find(1);
        // echo "<br>". $shop->age;//ตัวแปรshop ถูกjoinกับ userแล้วจึงสามารถเรียก dataใน userได้
        // ---------------------------------------
        // $shop =Shopmod::find(1);
        // echo "<br /> Shop :".$shop->name;//ชื่อ shop name
        // echo "<br /> name user ที่เป็นเจ้าของshop1:".$shop->user->name; //ชื่อเจ้าของ shop

        //-----------------------------------------------
        // $mod = usermod::find(1);//หาusermodที่1
        // echo "name :".$mod->name."<br/><br/>";
        // --------------------------------
        // $mods = usermod::find([1,3,4]);
        // foreach ($mods as $itemxx) {
        //     echo $itemxx->name." Email : ".$itemxx->email." Age: ".$itemxx->age."<br /><br />";
        //     //ตัวแปรจำนวน($itemxx)-> ชื่อข้อมูลในtableที่ต้องการโชว์
        // }
        // // --------------------------------
        // echo "name :".$mod->name;
      
        // $max = UserMod::where('active', 1)->max('age');
        // echo " age :".$max;

        //   $count = UserMod::where('active', 1)->count();
        // echo "จำนวน active ที่ = 1 :".$count;
*/